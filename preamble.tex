\usepackage[utf8]{inputenc}
\usepackage{amsmath, amsfonts, amssymb}

\usepackage{enumitem} % to control the manual enumerations dynamically

\usepackage{multirow}
\usepackage{multicol}
\usepackage{longtable}
\usepackage{booktabs}
\usepackage{needspace} %to make sure that no pagebreaks in certain situations
\usepackage{setspace}
\usepackage{blindtext}
%%%%%%%------------------------------------------------
%%%%%%-----------Page Layout Setups------
%%%%%%--------------------------------------------------
\usepackage{geometry}
\geometry{
	headheight=6ex,
	includehead,
	includefoot
}
\geometry{
	paper=a4paper,% Change to letterpaper for US letter
	inner=1in, % Inner margin
	outer=1in, % Outer margin
	bindingoffset=1cm, % Binding offset
	top=1in, % Top margin
	bottom=1in, % Bottom margin
	%showframe,% show how the type block is set on the page
}
\renewcommand{\baselinestretch}{1.3} % for line spacing
\usepackage{fancyhdr}%for the header and footer
\usepackage{emptypage}%to remove headers and footers in the blank pages	

\pagestyle{fancy} %Initiates the Page Style

\fancyhf{}
\renewcommand{\headrulewidth}{1pt}
%\fancyhead[L]{\nouppercase \leftmark}
\fancyhead[R]{\nouppercase \rightmark}
\renewcommand{\footrulewidth}{0.7pt}
\fancyfoot[L]{\small \em \printtitle}
\fancyfoot[R]{\thepage}




%%%%%%-----------------------------------------------
%%%%%%------------ Font Set up --------------
%%%%%%-----------------------------------------------
%\usepackage{fouriernc}
%\usepackage{newtxtext,newtxmath}
%\usepackage[T1]{fontenc} 
%\usepackage{mathdesign}
%\usepackage{mathptmx}
%\usepackage{utopia}
\usepackage{times} %Use this for Times New Roman
%\usepackage{helvet}
%\usepackage{lmodern}

%%%%%%------------------------------------------------
% % % %-----------------For the graphics-------
%%%%%%------------------------------------------------
\usepackage{graphicx}
\usepackage{subcaption}
\graphicspath{{./images/}}
\usepackage[usenames,dvipsnames,svgnames,table]{xcolor} % for colors
\usepackage{float}
\usepackage{tikz}
\usepackage[many,breakable]{tcolorbox}

%%%% Highlighter-----------------------------------------------
\makeatletter
\newcommand{\hl}[1]{%
	\setbox\@tempboxa\hbox{#1}%
	\ifdim\wd\@tempboxa>\linewidth
	\noindent
	\colorbox{pink}{%
		\parbox{\dimexpr\linewidth-2\fboxsep}{#1}%
	}%
	\else
	\colorbox{pink}{#1}%
	\fi}%Highlighter.
\makeatother
% % % % % % % %---------------------------------------
% % % % % % % %-------BibLatex------------------------
% % % % % % % %---------------------------------------
\usepackage[backend=bibtex,maxbibnames=6,defernumbers=true,natbib=true, sorting=none,  style = ieee,  url = false, doi = false]{biblatex}
\addbibresource{main.bib}



%%%%%------------------------------------------------
%%%%%-----------Hyperlinks Setup---------------------
%%%%%------------------------------------------------

\usepackage{hyperref}
\hypersetup{
	colorlinks=true,
	linkcolor=sec,
	filecolor=red,      
	urlcolor=sec,
	citecolor = sec
}

%%%%%------------------------------------------------
%%%%%-----------Title format-------------------------
%%%%%------------------------------------------------

\usepackage{etoolbox}
%\makeatletter
%\patchcmd{\chapter}{\if@openright\cleardoublepage\else\clearpage\fi}{}{}{}
%\makeatother

\newcommand{\Hrule}{\rule{\linewidth}{1mm}}



\usepackage[explicit]{titlesec}
\usepackage[titletoc]{appendix}
\definecolor{sec}{HTML}{154360}
\definecolor{band}{HTML}{EE9C52}


\makeatletter
\newcommand{\setappendix}{Appendix~\thechapter}
\newcommand{\setchapter}{Chapter~\thechapter~} 
\titleformat{\chapter}[display]{\sffamily\bfseries \huge\color{sec}}{}{0.5ex} %
{\ifnum\pdfstrcmp{\@currenvir}{appendices}=0
	\setappendix
	\else
	\setchapter
	\fi 
	\filleft \\%
	\titlerule[1pt]
	\vspace{1ex}%
#1\filleft\\
	\vspace{1ex}%
	\titlerule[3pt]}

\titleformat{name=\chapter, numberless}[display]{ \color{sec}\sffamily\bfseries\huge}{ }{0.5ex} %
{ \titlerule[1pt]
	\vspace{1ex}%
	#1\filleft
	\addcontentsline{toc}{chapter}{#1}
	\markboth{}{#1}\\
	\vspace{1ex}%
	\titlerule[3pt]}
\makeatother

\titlespacing*{\chapter}{0ex}{-10ex}{0ex}

\titleformat{\section}{\Large\bfseries}{}{0.0ex} %
{\sffamily\color{sec} \thesection~#1 \filright}[\smallskip]

\titleformat{name = \section, numberless}{\Large\bfseries}{}{0.0ex}%
{\sffamily\color{sec}#1\filright}[\smallskip]

\titleformat{\subsection}[hang]{\sffamily\Large\bfseries}{}{0.5ex}%
{\color{sec}\thesubsection~#1\filright}[\smallskip]

\titleformat{name = \subsection, numberless}[hang] {\sffamily\Large\bfseries}{}{0.5ex}%
{\color{sec}#1\filright}[\smallskip] 



%%%%%%%%%---------------------------------------------	
%%%%%%%%%----------New Theorems & Environments-------------------
%%%%%%%%%---------------------------------------------
\usepackage{amsthm}
\usepackage{thmtools} % handles autoref efficiently
\newtheorem{theorem}{Theorem}[chapter]  
\newtheorem{lemma}[theorem]{Lemma}%[section]
\newtheorem{corollary}[theorem]{Corollary}%[section]
\newtheorem{proposition}[theorem]{Proposition}%[section]




\theoremstyle{definition}
\newtheorem{question}{Question}
\newtheorem{conjecture}{Conjecture}
\newtheorem{example}[theorem]{Example}
\newtheorem{recall}{Recall}
\newtheorem{remark}[theorem]{Remark}
\newtheorem{definition}[theorem]{Definition}


\numberwithin{equation}{chapter}
\numberwithin{figure}{chapter}
\numberwithin{table}{chapter}


\newenvironment{keywords}{\noindent\textbf{Keywords:}}{}
\newenvironment{classification}{\noindent\textbf{AMS subject classifications:}}{}
%%%%%%%%%----------------------------------------------	
%%%%%%%%%----------for the Abstract environment:-------
%%%%%%%%%----------------------------------------------
\newcommand\abstractname{Abstract}  %%% here
\makeatletter
\newenvironment{abstract}{%
	\if@twocolumn
	\section*{\abstractname}%
	\else
	\small
	\begin{center}%
		{\bfseries \abstractname\vspace{-.5em}}%
	\end{center}%
	\quotation
	\fi}
{\if@twocolumn\else\endquotation\fi}
%	\fi
\makeatother
%	\listfiles %%To check the versions of the packages used

%----------------------------------------------------------------------------------------
%	PENALTIES( for the Hypenation Problem)
%----------------------------------------------------------------------------------------

\doublehyphendemerits=10000 % No consecutive line hyphens
\brokenpenalty=10000 % No broken words across columns/pages
\widowpenalty=9999 % Almost no widows at bottom of page
\clubpenalty=9999 % Almost no orphans at top of page
\interfootnotelinepenalty=9999 % Almost never break footnotes

%-----------------------------------------------------
%%Rulers used in Title page and other pages-----------
%-----------------------------------------------------
\newcommand{\decoRule}{\rule{.8\textwidth}{.4pt}}
\newcommand{\HRule}{\rule{\linewidth}{0.5mm}}

%%%% -----------------------------------------------------
%%%%--------- Index ---------------------------------
%%%% -----------------------------------------------------
\usepackage{imakeidx}
\makeindex[title={Subject Index},columns=1]
\makeindex[name=a,title=\uppercase{Author Index},columns=2]
\usepackage[font=small]{idxlayout}

\makeatletter
\newcommand\thesistitle[1]{\renewcommand\@thesistitle{#1}}
\newcommand\@thesistitle{} % To keep default empty
\newcommand{\printtitle}{\makeatletter \@thesistitle \makeatother} %to print the title wherever you need
\makeatother



\newcommand\dedicatory[1]{
	\clearpage 
	\null\vfil
	\thispagestyle{empty}
	\begin{center}{\Large\slshape #1}\end{center}
	\vfil\null
}
